\chapter{Introduction générale}


Nous allons vous présenter un compte rendu des recherches et travaux que nous avons réalisé durant le semestre de printemps 2015 dans le cadre de notre UV de TX. Notre objectif était de développer une bibliothèque de fonctions en langage C de cryptographie. Afin de traiter cette problématique nous avons du faire des rappels arithmétique sur les entiers (anneaux et corps sur les entiers) sur les polynômes (anneaux des polynôme), sur les corps de Galois et sur les structure algébriques associées (anneau/corps/anneau cyclique et élément générateur). 
Nous avons pu par la suite étudié les fonctions arithmétiques <<~de bases~>> comprenant les algorithmes d'Euclide étendu, d'exponentiation rapide, les calculs dans un corps de Galois et les tests de primalité des entiers. Nous avons aussi étudié les opération dans un corps de Galois et la recherche d'éléments générateur d'un corps cyclique. Et enfin, nous avons pu étudier les fonction de chiffrement et ainsi les implémenter au sein d’une bibliothèque en C++.\\


Mais avant de commencer, demandons nous qu’est-ce que la cryptographie. La cryptographie est une discipline s’attachant protéger des messages en assurant leur confidentialité, leur authenticité et leur intégrité. Ainsi, Kerckhoffs écrivait dans le journal des sciences militaires en 1883 un principe qui devint une des bases de la cryptographie :  <<~La sécurité d’un cryptosystème ne doit reposer que sur le secret de la clef. Autrement dit, tous les autres paramètres doivent être supposés publiquement connus.~>> Ce principe indique que l’on ne doit pas garder secret la manière de chiffrer l’information. Au contraire, le seul paramètre inconnu doit être la clef en elle même. Prenons pour exemple un système sécurisé qui constituerai un cadenas à code. Le principe de Kerckhoffs impliquerai qu’en rendant le mécanisme interne du cadenas publique, des experts et autres curieux pourraient tenter de trouver une faille permettant soit de retrouver la combinaison soit de forcer le cadenas. Ainsi le fabricant du cadenas pourrait corriger ces défauts jusqu’à obtenir un cadenas dont le mécanisme est connus de tous mais qui serait pourtant inviolable. Aujourd’hui, l’algorithme le plus utilisé dans l’échange sécurisé d’informations sur internet, l’algorithme RSA, fonctionne sur ce principe. Tout le monde connait son fonctionnement mais personne n’a trouvé de faille permettant de casser le chiffrement dans une période raisonnable. Ainsi dans ce genre de chiffrement, seul la clé secrète est l’élément important.\\

Nous avons aborder un peu plus haut la question de clé. En cryptographie, une clé peut être un mot, une phrase, une suite d’opération ou autre, qui sert de paramètre d’entrée à un algorithme de chiffrement ou de déchiffrement. Dans la cryptographie moderne, la clé est généralement constituée d’un grand nombre dont la longueur croissante assure une croissance sur la sécurité du message chiffré. Il existe deux types de clé : les clé publiques et les clefs privées. La distinction est simple, une clé publique est une clé qui est connue de tout le monde, son propriétaire va même surement s’occuper lui-même de la publier. La clé privée au contraire est un secret bien gardé par son propriétaire. Elle ne doit en aucun cas être retrouvée par une autre personne. Ces deux types de clé ne sont pas tout le temps employées. En effet il existe deux grandes manières de transmettre des informations chiffrées. Utiliser un système de cryptographie symétriques ou asymétrique. La cryptographie symétrique se base sur une clé secrète qui est connue par les deux personnes qui communiquent. Dans le cas de l’asymétrique, on utilise deux clés. Une clé publique va servir à chiffrer les messages et une clé privé sert à décoder ces messages. Seulement le problème avec la cryptographie symétrique c’est qu’il faut se communiquer la clé secrète sans que quelqu’un d’autre puisse l’intercepter. C’est pourquoi on utilise généralement la cryptographie asymétrique dans un premier temps pour convenir d’une clé symétrique que l’on utilisera par la suite. Pour illustrer ce principe, prenons un exemple. Imaginons que Bob et Alice souhaitent s’échanger des informations confidentielles. Alice va générer un couple clé privé, clé publique à l’aide d’un algorithme ou d’une fonction. Elle va ensuite envoyer à Bob la clé publique. Ici, il n’y a aucune crainte quant à la sécurité du message contenant la clé publique car si quelqu’un d’autre obtenait cette clé il ne pourrait qu’envoyer des messages codés à Alice et rien de plus. Par la suite Bob va envoyer une clé commune en la chiffrant grâce à la clé publique d’Alice. Le processus n’est pas inversible avec la clé publique. Ainsi Bob peut envoyer le message à Alice sans craindre que quelqu’un l’intercepte. Une fois ce message chiffré reçu, Alice peut le déchiffrer grâce à sa clé privée. Elle obtient donc la clé commune que Bob lui a envoyé. Par la suite Alice et Bob utiliseront cette clé commune pour chiffrer et déchiffrer les messages qu’ils s’enverront. Cette manière de faire est utilisée partout aujourd’hui dans le monde de l’informatique lorsque l’on veut établir une connexion sécurisée. Cependant il existe différents algorithme pour faire de l’asymétrique ou du symétrique. Ils reposent sur des fonctions mathématiques et des résultats obtenus par des mathématiciens au cours du temps. Nous allons détailler par la suite les différents algorithmes étudiés pour construire notre bibliothèque.

%TODO : Ahérer un peu
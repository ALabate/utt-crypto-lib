\documentclass[t]{beamer}
\usepackage[utf8]{inputenc}
\usepackage[francais]{babel}
\usepackage{tikz,pgf}
\usepackage{wrapfig}
\usepackage{times}
\usepackage[absolute,overlay]{textpos}
\usepackage{url}
% \usepackage[texcoord,grid,gridunit=mm,gridcolor=red!10,subgridcolor=green!10]{eso-pic}



% Display code
\usepackage{listings}
    \definecolor{dkgreen}{rgb}{0,0.6,0}
    \definecolor{gray}{rgb}{0.5,0.5,0.5}
    \definecolor{mallow}{rgb}{0.58,0,0.82}

    \lstset{
        language=C++,
        aboveskip=3mm,
        belowskip=3mm,
        showstringspaces=false,
        columns=flexible,
        basicstyle={\small\ttfamily},
        numbers=none,
        numberstyle=\tiny\color{gray},
        keywordstyle=\color{blue},
        commentstyle=\color{dkgreen},
        stringstyle=\color{mallow},
        breaklines=true,
        breakatwhitespace=true,
        tabsize=4,
        frame=single,
        literate=%
            {é}{{\'e}}1
            {É}{{\'E}}1
            {è}{{\`e}}1
            {à}{{\`a}}1
            {È}{{\`E}}1
            {À}{{\`A}}1
            {ê}{{\^e}}1
            {â}{{\^a}}1
            {î}{{\^\i}}1
            {ô}{{\^o}}1
            {û}{{\^u}}1
            {Ê}{{\^E}}1
            {Â}{{\^A}}1
            {Î}{{\^I}}1
            {Ô}{{\^O}}1
            {Û}{{\^U}}1
            {ë}{{\"e}}1
            {ï}{{\"\i}}1
            {ü}{{\"u}}1
            {Ë}{{\"E}}1
            {Ï}{{\"I}}1
            {Ü}{{\"U}}1
            {ç}{{\c c}}1
            {Ç}{{\c C}}1
            {æ}{{\ae}}1
            {Æ}{{\AE}}1
            {œ}{{\oe}}1
            {Œ}{{\OE}}1
            {Å}{{\AA}}1
    }




\definecolor{bleu}{rgb}{0.282, 0.439, 0.643}

\usecolortheme[named=bleu]{structure}
\useinnertheme[shadow]{rounded}

\hypersetup{pdfpagemode=FullScreen}

\setbeamersize{text margin left=0.6cm}
\setbeamersize{text margin right=0.4cm}
\setbeamersize{text margin left=.5cm, text margin right=.5cm}
\setbeamertemplate{navigation symbols}{}

\AtBeginSection[]{
    \begin{frame}
  	 \frametitle{\vspace*{-1cm} \begin{flushright} \ \ \vskip-0.25cm \end{flushright} }
        \vspace*{-0.75cm}
	    \begin{center}
            \centering
            \tableofcontents[currentsection]
	    \end{center}
    \end{frame}
}

\begin{document}
    \begin{frame}
        \pgfputat{\pgfxy(-0.505,0.685)}{\pgfbox[left,top]{\pgfimage[width=\paperwidth,height=\paperheight ]{./assets/UTT_garde_empty.pdf}}}
        \vspace*{2.5cm}

        \begin{center}
            \textcolor{white}{
                {\large Soutenance de TX} \\
                {\footnotesize   Printemps 2015}\\
            \vspace*{0.5cm}
                {\footnotesize \textbf{Développement d'une de fonctions de chiffrement en C}}\\
            \vspace*{0.25cm}
                {\footnotesize Pergoud Florent\\ Labate Aurélien}\\
          }
        \end{center}
    \end{frame}

\setbeamertemplate{background canvas}{
    \includegraphics[width=\paperwidth,height=\paperheight]{./assets/UTT_fond.pdf}}
    \addtobeamertemplate{footline}{
    \vspace{-0.5cm}
    \begin{flushright}
    \insertframenumber/\inserttotalframenumber
    \end{flushright}
}


\section{Introduction}
    \begin{frame}[c]
        \begin{center}
            \frametitle{\vspace*{-1.0cm} \begin{flushright} Introduction \ \ \vskip-0.25cm \end{flushright} }
            \vspace*{-0.75cm}
            \begin{itemize}
                \item Développement d'un robot autonome pour la coupe de France
                \item Classement: 54èmes sur plus de 180 équipes
                \item Casse mécanique du robot principal !
                \item Objectif: proposer des bases théoriques, algorithmiques et logicielles solides pour les années suivantes
            \end{itemize}
        \end{center}
    \end{frame}

    \begin{frame}[c]
        \frametitle{\vspace*{-1.0cm} \begin{flushright} Introduction \ \ \vskip-0.25cm \end{flushright} }
        La robotique est une discipline complexe, elle allie:
        \begin{center}
            \begin{itemize}
                \item Informatique
                \item Électronique
                \item Mécanique
                \item Sciences physiques, algorithmique..
            \end{itemize}
        \end{center}
        Dans cette présentation, on parlera d'\textbf{informatique} et de \textbf{réseau}.
    \end{frame}


\section{Environnement de programmation}


\defverbatim[colored]\makeset{\begin{lstlisting}[language=C++]
    mpz_t a, b, c, res;
    mpz_init_set_ui(a, 42);
    mpz_init_set_str(b, "3172371623812", 10);
    mpz_init_set_str(c, "1A", 16);

    mpz_mul(res, b, c);
    mpz_add(res, a, res);

    printf("%s\n", mpz_get_str (NULL, 10, res));
\end{lstlisting}}

    \begin{frame}[c]
        \frametitle{\vspace*{-1.0cm} \begin{flushright}Langage de programmation \ \ \vskip-0.25cm \end{flushright} }
        \begin{itemize}
            \item Projet originalement proposé en C avec GMP
            \item Mais la syntaxe est très lourde en utilisant la librairie GMP en C
            \item Par exemple pour $42+ 3172371623812\times 0x1A$ :
        \end{itemize}
        \makeset
    \end{frame}


\defverbatim[colored]\makeset{\begin{lstlisting}[language=C++]
    mpz_class a, b, c, res;
    a = 42;
    b = "3172371623812";
    c = "0x1A";

    res = a + b * c;

    std::cout << "Résultat : " << res << std::endl;
\end{lstlisting}}


    \begin{frame}[c]
        \frametitle{\vspace*{-1.0cm} \begin{flushright}Langage de programmation \ \ \vskip-0.25cm \end{flushright} }
        \begin{itemize}
            \item Alors qu'avec la version C++ de GMP :
            \makeset
            \item Cependant, le C++ est moins connu est peut gêner
        \end{itemize}
    \end{frame}


\defverbatim[colored]\makeset{\begin{lstlisting}[language=C++]
    mpz_class a, res;
    a = 2;

    res = mpz_pow_ui(a.get_mpz_t(), 3);

    std::cout << "Résultat : " << res << std::endl;
\end{lstlisting}}


    \begin{frame}[c]
        \frametitle{\vspace*{-1.0cm} \begin{flushright}Bibliotheque \ \ \vskip-0.25cm \end{flushright} }
        \begin{itemize}
            \item La version C++ de GMP manque certaines fonctionnalités
            \item Ainsi pour calculer $2^3$ :

            \makeset
            \item On doit utiliser les fonction de la version C
        \end{itemize}
    \end{frame}



\defverbatim[colored]\makeset{\begin{lstlisting}[language=C++]
    mpz2_class a, res;
    a = 2;

    res = a.pow(3);

    std::cout << "Résultat : " << res << std::endl;
\end{lstlisting}}


    \begin{frame}[c]
        \frametitle{\vspace*{-1.0cm} \begin{flushright}Bibliotheque \ \ \vskip-0.25cm \end{flushright} }
        \begin{itemize}
            \item Nous avons donc créer une librairie qui évite cela tout en gardant les avantages :

            \makeset
        \end{itemize}
    \end{frame}


    \begin{frame}[c]
        \frametitle{\vspace*{-1.0cm} \begin{flushright} Architecture modulaire \ \ \vskip-0.25cm \end{flushright} }
        \vspace*{1.5cm}
        \begin{center}
            \textbf{Le coeur}
        \end{center}
        \begin{center}
            \begin{itemize}
                \item Commande tous les autres modules
                \item Plus de puissance de calcul
                \item Language plus haut niveau
            \end{itemize}
        \end{center}

        Les autres modules :
        \begin{center}
            \begin{itemize}
                \item Moteur
                \item Pince
                \item Capteurs
            \end{itemize}
        \end{center}
    \end{frame}
    \begin{frame}[c]
        \frametitle{\vspace*{-1.0cm} \begin{flushright} Architecture modulaire \ \ \vskip-0.25cm \end{flushright} }
        \vspace*{1.5cm}
        \begin{center}
            \textbf{Module moteur}
        \end{center}
        \begin{center}
            \begin{itemize}
                \item Récupère les informations des roues codeuses
                \item Commande et asservis les moteurs
                \item Calcul la position et la trajectoire
            \end{itemize}
        \end{center}
    \end{frame}
    \begin{frame}[c]
        \frametitle{\vspace*{-1.0cm} \begin{flushright} Architecture modulaire \ \ \vskip-0.25cm \end{flushright} }
        \vspace*{1.5cm}
        \begin{center}
            \textbf{Module pince}
        \end{center}
        \begin{center}
            \begin{itemize}
                \item Génère les impulsions à fréquence optimale pour controller les moteurs
                \item Initialise la position des moteurs pas à pas
            \end{itemize}
        \end{center}
    \end{frame}
\section{Les algorithmes principaux}
        \begin{frame}[c]
            \frametitle{\vspace*{-1.0cm} \begin{flushright} Les algorithmes principaux \ \ \vskip-0.25cm \end{flushright} }
            \vspace*{0.5cm}
			\begin{center}
			\textbf {Les différents algorithmes}
			\end{center}			            
             \begin{textblock*}{1cm}(4cm,3cm)
                \textbf{UART}
            \end{textblock*}
            \begin{center}
                \begin{itemize}
                    \item Liaison full-duplex
                    \item Niveau logique: TTL
                    \item Connexion asynchrone
                    \item CRC basique: bit de parité
                    \item Débit (baudrate) variable, maximum 3,6GBauds théoriques
                    \item 2 fils par pair de périphérique
                \end{itemize}
            \end{center}
        \end{frame}

        \begin{frame}[c]
            \frametitle{\vspace*{-1.0cm} \begin{flushright} Technologies de communication \ \ \vskip-0.25cm \end{flushright} }
            \vspace*{0.5cm}
             \begin{textblock*}{1cm}(4cm,3cm)
                \textbf{SPI}
            \end{textblock*}
            \begin{center}
                \begin{itemize}
                    \item Liaison full-duplex
                    \item Bus de donnée, relation master/slave
                    \item Connexion synchrone: le master donne la clock
                    \item Bus de donnée de deux fil, et un fil d'addressage par esclave
                    \item Débit élevé pour un bus, jusqu'à 20Mb/s
                    \item Protocle simple, pas d'adressage/gestion d'erreur
                \end{itemize}
            \end{center}
        \end{frame}

        \begin{frame}[c]
            \frametitle{\vspace*{-1.0cm} \begin{flushright} Technologies de communication \ \ \vskip-0.25cm \end{flushright} }
            \vspace*{0.5cm}
             \begin{textblock*}{1cm}(4cm,3cm)
                \textbf{CAN}
            \end{textblock*}
            \begin{center}
                \begin{itemize}
                    \item Liaison half-duplex
                    \item Bus de donnée, relation multi-master
                    \item Connexion asynchrone
                    \item Detection et retour sur erreur physique
                    \item Débit moyen pour un bus, jusqu'à 1Mb/s
                    \item Fonctionnement "événementiel"
                    \item Jusqu'à 30 périphériques, système d'adressage
                    \item Très résistant aux perturbations
                \end{itemize}
            \end{center}
        \end{frame}

        \begin{frame}[c]
            \frametitle{\vspace*{-1.0cm} \begin{flushright} Technologies de communication \ \ \vskip-0.25cm \end{flushright} }
            \vspace*{0.5cm}
             \begin{textblock*}{1cm}(4cm,3cm)
                \textbf{I2C}
            \end{textblock*}
            \begin{center}
                \begin{itemize}
                    \item Liaison half-duplex
                    \item Bus de donnée, relation master/slave ou multi-master
                    \item Connexion synchrone: un fil commun de clock
                    \item Débit faible, jusqu'à 400Kb/s
                    \item Jusqu'à une centaine de périphériques, système d'adressage
                    \item Un bus de donnée sur un fil
                    \item Aquittement des données
                \end{itemize}
            \end{center}
        \end{frame}

        \begin{frame}[c]
            \frametitle{\vspace*{-1.0cm} \begin{flushright} Technologies de communication \ \ \vskip-0.25cm \end{flushright} }
            \vspace*{0.5cm}
             \begin{textblock*}{1cm}(4cm,3cm)
            \end{textblock*}
            Technologie choisie: I2C.\\
            Problèmes:
            \begin{center}
                \begin{itemize}
                    \item Intégrité des données
                    \item Architecture master/slave
                    \item Librairies Javascript peu abouties
                \end{itemize}
            \end{center}
        \end{frame}

        \begin{frame}[c]
            \frametitle{\vspace*{-1.0cm} \begin{flushright} Technologies de communication \ \ \vskip-0.25cm \end{flushright} }
            \vspace*{0.5cm}
            \textbf{Construction d'une surcouche du protocole}\\
            \ \\
            Le but principal est d'avoir un retour sur erreur.
            \begin{center}
                \begin{itemize}
                    \item Taille du packet annoncée
                    \item Numéro de packet
                    \item Données
                    \item CRC: XORSum
                \end{itemize}
            \end{center}
            TABLEAU - FRAME ici
        \end{frame}

\section{Conclusion}
    \begin{frame}[c]
        \frametitle{\vspace*{-1.0cm} \begin{flushright} Conclusion \ \ \vskip-0.25cm \end{flushright} }
        \vspace*{-0.75cm}
        \begin{itemize}
            \item Travail important sur les parties réutilisables
            \item Une documentation interne au club
            \item Le travail continu :
            \begin{itemize}
                \item Le bus can
                \item La modularité jusqu'au physique
                \item Des Arduino faites maison
                \item Moteur brushless
            \end{itemize}
            \item Un robot que nous pour les prochains évènements
        \end{itemize}

    \end{frame}



\end{document}